\documentclass[../dissertation.tex]{subfiles}

\begin{document}

\section{Experiment 8 - Vertical Scaling On Car Platform}
\label{experiment8-vertical-scaling}

Prior experiments have been completed using what was called `bare' Raspberry Pi hosts. This refers to the fact that they are off-the-shelf Raspberry Pis with no peripherals or modifications, using standard mains power. However, the test platform described in Section \ref{background-robot-config} represents a more realistic scenario, utilising a more unreliable power source (a battery), and several peripherals (such as a video camera). These differences in the host configuration are not expected to affect performance of the vertical scaling limits in the host (since the car platform utilises the same model of Raspberry Pi) - however, this must be experimentally confirmed.

As such, this experiment consists of a re-run of Experiment 7 (as presented in Section \ref{experiment7-vertical-scaling}), but on the robot car platform. This presents some new challenges in the execution of the experiment. Due to the battery power supply the entire experiment (3 runs at 3 message frequencies with 8 node counts) can not be run in a single battery lifetime. A single charge lasts long enough for 1.5 runs, however in order to reduce liklihood of incorrect data it was decided to conduct the experiment 1 run at a time, with a battery charge conducted between runs.

\subsection{Results}





\end{document}
