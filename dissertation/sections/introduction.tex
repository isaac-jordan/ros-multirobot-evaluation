\documentclass[../dissertation.tex]{subfiles}

\begin{document}

\chapter{Introduction}
\label{introduction-chapter}

\section{Context}

With the increase in availability of commodity computing hardware it is becoming more feasible to experiment with robotics at home. This increase in hobbyist roboticists is one plausible reason for the increase in open source robotics development. These community driven development projects are generally very flexible and extensible, given the large number of types of users wishing to use them in different ways: students, hobbyists, researchers, and enterprises. One such project is ROS (Robot Operating System). ROS describes itself as ``a collection of tools, libraries, and conventions that aim to simplify the task of creating complex and robust robot[s]''\cite{rosaboutpage}. Much of the knowledge on ROS and it's performance is anecdotal acquired and shared on sites such as StackOverflow, this project aims to present a more formal analysis of the performance and scalability of ROS both in an isolated system, and in a real robot car kit platform.

A system created using ROS comprises a number of nodes, where each node is conceptually a self-contained module with the goal of completing a single task. The behaviour of a robot which is built using ROS comes as an emergent property of the collection of nodes forming the robot. ROS nodes need not be colocated on a single machine (called a host). This allows for nodes to be distributed over a number of hosts; in the case that each host is an indepedent robot this creates a multi-robot system. The robot car kit platform used in this project comprises a number of Sunfounder Smart Video Car Kits\cite{SunfounderRobotCarKit}, with each car being driven by a Raspberry Pi 3 Model B. The high availability of cheap self-contained computing devices such as Raspberry Pis means that these cat kits are representative of what many roboticists might use to drive their robots.

Scalability is a term widely used to refer to many different ideas, however this project uses the term to refer to three different types of scalability. The first, called frequency scaling, involves increasing the message frequency from a particular ROS node - until the system can no longer process the messages in a timely manner. The second, vertical scaling, refers to adding multiple ROS nodes to the same host. The third, horizontal scaling, involves increasing the number of hosts in the system while possibly increasing the number of nodes (depending on whether per-host node count or overall node count should be kept constant).

\section{Aims and Objectives}

This project aims to provide an evaluation of ROS in a multi-robot situation, by first evaluating the simplest multi-robot case (2 robots/hosts communicating with each other, 1 ROS node each) and looking at the limits of communication in this scenario - and attempt to identify possible causes for the limits. The next aim is to vertically scale these hosts to run many more nodes each - in order to explore the upper bound of how much communication a single ROS host can sustain (whether in terms of messages-per-second, bytes-per-second, or number of sending/receiving processes).

The intention of the research is to build a foundation of knowledge upon which further research in to communication systems of ROS can be conducted, so that future researchers need not depend on anecdotal performance estimations.

\section{Achievements}

This project makes several key contributions:

\begin{itemize}

  \item \textbf{A comprehensive review of existing robotic middlewares, including an in-depth analysis of ROS} A survey of a wide range of software packages that compete with ROS as robot middleware was conducted, providing an understanding of where ROS is placed in the market. A number of common themes are identified across the variety of middlewares, broken down in terms of communication, computation, configuration, and coordination. This is covered in Chapter \ref{background-chapter}.

  \item \textbf{A systematic evaluation of high frequency communication performance with ROS in a multi-host network} A sequence of experiments were conducted investigating ROS' ability to scale to high frequency messages with inter-host communication. Frequencies ranging from 1Hz to 1MHz were considered with emphasis in the 200Hz to 2KHz range, with message sizes ranging from 11 bytes (a simple string), to 4KB sensor readings (recorded data provided by the MIT Stata Center dataset\cite{mit-stata-center-dataset}) as well as 300KB video image messages (from the same dataset). Key findings are that high frequency ROS communication was constrained by processing power of the test system (a 50\% reduction in CPU clock speed doubles message latency), however this reduces in importance as message sizes increase. Usage of a Wi-Fi communication channel was found to introduce significant latency compared to Ethernet (5ms latency at 2KHz with Ethernet, and 1800ms when using Wi-Fi), and reduce the maximum frequency that ROS could send messages at. These findings were confirmed to hold true on the Sunfounder robot kit car, indicating the results are applicable to real robot platforms. These experiments are covered in Chapter \ref{communication-chapter}.

  \item \textbf{Proposal of a Communication Scaling Limit Volume (CSLV) which predicts frequency and vertical scalability limits} A trend in communication performance led to the proposal of a CSLV formula, allowing for the performance of a particular ROS host to be calculated given the total number of nodes on the host, their message frequencies, and the individual message size. To verify the predictive capabilities of the CSLV, two series of vertical scaling experiments were conducted on a pair of hosts. The first set of experiments used a range of high message frequencies (100Hz to 300Hz) to calculate an approximate CSLV value for the Raspberry Pi 3 Model 3 system. The next series of experiments were conducting using a lower range of frequencies (1Hz to 20Hz), and the performance results were correctly predicted by the CSLV formula - establishing the CSLV hypothesis. These experiments were conducted using 4KB per message sensor data acquired from the MIT Stata Center dataset\cite{mit-stata-center-dataset}, and were also repeated on a pair of Sunfounder robot car kit hosts. An experimental set-up is also proposed to conduct horizontal scaling experiments. These results are covered in Chapter \ref{host-scalability-chapter}.

\end{itemize}

\end{document}
