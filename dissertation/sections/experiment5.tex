\documentclass[../dissertation.tex]{subfiles}

\begin{document}

\subsection{Experiment 5 - CPU Clock Speed (Real Data)}

This experiment aims to verify the results of Experiment 3 in Section \ref{experiment3-cpu-speed} when using real data. This is achieved by repeating the experimental set-up and procedure, but replacing the sent message with a recorded payload.

The experiment was repeated using the previously mentioned sensor data, as well as video data.

The expectation was that sensor data (with it's relatively small message sizes) would give similar results to the dummy data used previously (a string consisting of `hello world'), and that video data would demonstrate different performance characteristics due to the significantly different message sizes.

This is achieved using a ROS bag provided as part of the MIT Stata Center Dataset. A ROS bag is a datastructure which allows replaying of ROS topics. The topic can be replayed at the same rate it was recorded at, or any other desired rate. This allows for a variety of message frequencies to be used, as in previous experiments.

\subsubsection{Results}

The experiment appeared to confirm the results of Experiment 4. For sensor data, 100\% CPU speed (1.2GHz) was the lowest latency at 7 out of 10 message frequencies, as shown in Figure \ref{exp5-sensor-means-all-freq}, and 50\% CPU speed (600MHz) was slowest at 9 out 10 frequencies.

\begin{figure}[H]
\centering
\includegraphics[width=\textwidth]{images/experiment5/sensor_data_all_freqs.png}
\caption{Experiment 5 - Sensor Data, All Frequencies}
\label{exp5-sensor-means-all-freq}
\end{figure}

For video data, the trend held at the lower experimental frequencies (see Figure \ref{exp5-video-means-low-freq}) with low message latencies, which stepped up slightly when CPU speed was decreased. However, the change in performance was significantly less than for the smaller message sizes of the sensor data - for example, at 20Hz the top and bottom values were only 4.9\% different, whereas for sensor data at 400Hz the top and bottom values were 54\% different. This leads to the conclusion that CPU speed is less impactful as message sizes increase. CPU speed became entirely insignificant at higher message frequencies for video data, as it is thought the network interface becomes the bottleneck. Message frequencies at larger than 30Hz gave much higher average latencies (around 4 seconds at 40Hz, compared to 60 milliseconds at 20Hz), and showed no correlation with CPU speed (see Figure \ref{exp5-video-means-high-freq}.

\begin{figure}[H]
\centering
\includegraphics[width=\textwidth]{images/experiment5/video_data_low_freqs.png}
\caption{Experiment 5 - Video Data, Low Frequencies}
\label{exp5-video-means-low-freq}
\end{figure}

\begin{figure}[H]
\centering
\includegraphics[width=\textwidth]{images/experiment5/video_data_higher_freqs.png}
\caption{Experiment 5 - Video Data, High Frequencies}
\label{exp5-video-means-high-freq}
\end{figure}

\end{document}
