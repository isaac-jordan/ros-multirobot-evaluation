\pdfoutput=1

\documentclass{l4proj}

%
% put any packages here
%
\usepackage{float}
\usepackage{enumitem}
\usepackage{longtable}
\usepackage{listings}
\usepackage{hyperref}
\usepackage{subfiles}
\usepackage{color}
\usepackage{graphicx}

\newcommand{\nc}[1]{\textcolor{magenta}{(\textbf{NC:} #1)}}
\newcommand{\gac}[1]{\textcolor{cyan}{(\textbf{GAC:} #1)}}
\newcommand{\pwt}[1]{\textcolor{blue}{(\textbf{PWT:} #1)}}

\graphicspath{{images}{../images/}}

\lstdefinestyle{codestyle}{
    %basicstyle=\footnotesize,
    %breakatwhitespace=false,
    breaklines=true,
    %captionpos=b,
    %keepspaces=true,
    %numbers=left,
    %numbersep=5pt,
    %showspaces=false,
    %showstringspaces=false,
    showtabs=false,
    tabsize=2
}

\lstset{style=codestyle}

\begin{document}
\title{Evaluating the Scalability of ROS in Multi-Robot Systems}
\author{Isaac Jordan}
\date{\today}
\maketitle

\begin{abstract}
Autonomous robots are being created in an increasingly modular way. This is leading to the creation of many types of robot middleware software frameworks, however much of the existing knowledge about the performance of these middlewares is informal and anecdotal. Commodity hardware such as the Raspberry Pi is becoming more prevalent in robotic systems, providing a computation base with networking such as Wi-Fi and Ethernet. A single system comprising multiple individual autonomous robots forms a multi-robot system. These systems have added complexity over single robot systems as inter-robot communication can be orders of magnitude slower than intra-robot communication. High frequency communication between robots is desirable as sensor technology improves and larger amounts of data is collected in robotic systems.

A survey of a wide range of software packages that compete with ROS as robot middleware was conducted, providing an understanding of where ROS is placed in the market. Several common themes are identified across the variety of middlewares, broken down in terms of communication, computation, configuration, and coordination.

A sequence of experiments was conducted investigating ROS' ability to scale to high frequency messages with inter-host communication. Frequencies ranging from 1Hz to 1MHz were considered with emphasis in the 200Hz to 2KHz range, and message sizes ranging from 11 bytes (a simple string) to 4KB sensor readings (recorded data provided by the MIT Stata Center dataset) as well as 300KB video image messages (from the same dataset). Key findings are that high frequency ROS communication was constrained by processing power of the test system (a 50\% reduction in CPU clock speed doubles message latency), however this reduces in importance as message sizes increase. Usage of a Wi-Fi communication channel was found to introduce significant latency compared to Ethernet (5ms latency per message at 2KHz with Ethernet, and 1800ms when using Wi-Fi), and also reduced the maximum frequency that ROS could send messages at. These findings were confirmed to hold true on the Sunfounder robot kit car, indicating the results are applicable to real robot platforms.

A trend in communication performance led to the proposal of a Communication Scaling Limit Volume (CSLV) formula, allowing for the performance of a particular ROS host to be calculated given the total number of nodes on the host, their message frequencies, and the individual message size. To verify the predictive capabilities of the CSLV, two series of vertical scaling experiments were conducted on a pair of hosts. The first set of experiments used a range of high message frequencies (100Hz to 300Hz) to calculate an approximate CSLV value for the Raspberry Pi 3 Model B system. The next series of experiments were conducting using a lower range of frequencies (1Hz to 20Hz), and the performance results were correctly predicted by the CSLV formula - establishing the CSLV hypothesis. These experiments were conducted using 4KB per message sensor data acquired from the MIT Stata Center dataset, and were also repeated on a pair of Sunfounder robot car kit hosts. An experimental set-up is also proposed to conduct horizontal scaling experiments.

\end{abstract}

\educationalconsent
%
%NOTE: if you include the educationalconsent (above) and your project is graded an A then
%      it may be entered in the CS Hall of Fame
%
\tableofcontents
%==============================================================================

\pagebreak
\pagenumbering{arabic}

\subfile{sections/introduction}

\chapter{Background}
\label{background-chapter}

\subfile{sections/background-part1}

\subfile{sections/background-part2-middleware-overview}

\subfile{sections/background-part3-middleware-discussion}

\subfile{sections/background-part4}

\chapter{Factors Affecting High Frequency Message Transmission}
\label{communication-chapter}

Multi-robot systems are types of distributed systems. The main concern in moving from a single-host distributed system to a multi-host distributed system is the introduction of more complex communication network. As the overall system is trying to complete a task, the individual nodes in the network must communicate in order to share things like the status of the node (CPU load, disk usage, etc), progress through the current task, and results of a task. The exact nature and direction of this communication is dependent on the architecture of the distributed system.

In ROS, the Master node monitors the progress and status of the other nodes, but does not involve itself in the application logic of the system, meaning that it does not process or handle the results of nodes. This is handled in a peer-to-peer (P2P) fashion dictated by the application developer - each node is responsible for choosing what information to request from other nodes, and what information to emit from itself using the Publisher/Subscriber model described in Section \ref{background-ros}. The consequence of this architecture is that overall system performance can be dramatically affected by the communication performance between nodes.

In order to systematically evaluate how ROS' communication scales, one must first aim to understand which aspects of the system that are of greatest importance to performance - both in terms of latency of messages, and total throughput of communication. Section \ref{communication-scoping-experiments} aims to analyse the performance bottlenecks of a simple multi-robot system when sending very controlled and artificial data. Section \ref{communication-real-data} will then aim to validate the results of Section \ref{communication-scoping-experiments} using previously recorded data streams that are representative of a typical ROS system, with the goal that these conclusions will then be directly applicable to real ROS multi-robot systems. This chapter aims to provide an understanding of the fundamental communication performance characteristics of a multi-robot ROS system, so that vertical and horizontal scaling experiments can be contextualised.

Full datasets and extra graphs of all experiments can be found in the experiment repository\cite{experimentGitHubRepo}.

\begin{table}[H]
\centering
\caption{Chapter \ref{communication-chapter} Experiment Overview}
\label{communication-experiment-overview}
\begin{tabular}{|p{0.25\columnwidth}|p{0.4\columnwidth}|p{0.1\columnwidth}|}
\hline
\textbf{Experiment}         & \textbf{Description}                                                                              & \textbf{Section}              \\ \hline
Revising Existing Code      & Building upon prior work to obtain baseline expectations of ROS communication performance         & \ref{exp-1}                 \\ \hline
Impact of Rebooting        & Identifying whether rebooting the host machines affects performance of repeated message streams   & \ref{exp-2}                 \\ \hline
Impact of CPU Clock Speed  & Identifying whether performance of the message streams is limited by processing power             & \ref{experiment3-cpu-speed} \\ \hline
Impact of Wi-Fi Connection & Investigating the impact of performing ROS communication over a Wi-Fi based communication channel & \ref{exp-4}                 \\ \hline
Impact of CPU Clock Speed when using Representative Data & Aiming to validate the results of the CPU Clock Speed scoping experiment on data more typical of a ROS system  & \ref{exp-5} \\ \hline
Impact of Wi-Fi Connection when using Representative Data & Aiming to validate the results of the Wi-Fi Connection scoping experiment using data more typical of a ROS system & \ref{experiment-6} \\ \hline

\end{tabular}
\end{table}

\section{Scoping Experiments}
\label{communication-scoping-experiments}

In order to acquire a detailed understanding of the performance characteristics of ROS' communication channels, a sequence of scoping experiments was designed. The experiments begin very simply - trying to gain baseline expectations of how well ROS will perform, and where problems may occur. Further experiments then investigate individual experimental parameters to identify problem areas for ROS. These experiments were performed using highly controlled data - specifically a single string consisting of `Hello World'. This allows for rapid development and iteration of experiments, while keep the data sent constant. Later experiments (in Section \ref{communication-real-data}) then explore how varying the data sent affects the conclusions of the scoping experiments.

\subfile{sections/experiment1}

\subfile{sections/experiment2}

\subfile{sections/experiment3}

\subfile{sections/experiment4}

\section{Representative Data Experiments}
\label{communication-real-data}

Previous experiments (in Section \ref{communication-scoping-experiments}) were presented as scoping experiments. These were designed to systematically evaluate and gain understanding of which variables were of concern when running tests on ROS' communication performance. However, it was not certain that these prior conclusions would translate well to using ROS in a realistic scenario since throughout the scoping experiments, the same message of `hello world' was used as dummy data. This is a very small message payload of only 11 bytes.

In order to verify the conclusions were correct, samples of realistic data were explored. Two data types were settled on, `sensor data' and `video data'. Sensor data is the type of data likely the come from a physical sensor on the robot. This data is characterised by small message sizes, such as 4KB. Video data is the type of data recorded from a video camera peripheral, such as a webcam. This type of data generally has larger message sizes in range of hundreds of kilobytes, up to several megabytes per message depending on the resolution and bitrate of the camera.

The exact data sets used in the coming experiments were acquired from the MIT Stata Center dataset \cite{mit-stata-center-dataset}. The dataset is described as follows by the producers of the data: ``The MIT Stata Center Data Set is a vast scale data set collected over a multi-year period in a 10 storey academic building. It contains sensor data collected since January 2011. As of September 2012 the data set comprises over 2.3TB, 38 hours and 42 kilometres (the length of a marathon)''\cite{mit-stata-center-dataset}. A dataset from 2012 may appear outdated; however, it suits the goal of these experiments perfectly, as only data types and message sizes, which haven't changed since, are of importance. The data is recorded from sensors on the PR2 (a two-armed wheeled robot), which was one of the first robots to ever be designed using ROS as a middleware. The exact values of the data are not important for this evaluation of ROS, it is only the size and frequency of the messages that are important as there is no analysis or computation being executed on the data being transmitted.

This dataset contains both sensor feeds (from a laser sensor), and video feeds (from a Kinect RGB + depth camera). The sensor data was a 20Hz message stream, which was measured to use 85kB/s of bandwidth, giving an individual message size of 4.25kB. The video data used was a 30Hz (30 frames-per-second) RGB (colour) video stream, with a resolution of 640 x 480 pixels. This message stream was measured to use 9.25MB/s of bandwidth, implying a message size of 308KB. These data sets are very large (20 - 50GB) which would require modification to the test system (Raspberry Pi 3’s). Thus, these datasets have been filtered down to 60 seconds of recording, resulting in 1791 camera images, and 1194 LaserScan readings. This number of messages was chosen at it would give a good number of messages to average latency over (over 1000 for each data type), but would also easily fit on to the SD-card storage of the Raspberry Pis, which is only 8GB in size. The total size of the 60 seconds of data was approximately 1GB.

\subfile{sections/experiment5}

\subfile{sections/experiment6}

\chapter{Distributed Scalability with ROS}
\label{host-scalability-chapter}

Chapter \ref{communication-chapter} provided grounding work to build further experiments upon, with an understanding gained of ROS' communication performance at a variety of message frequencies, and with a variety of hardware set-ups such as reduced processing power and Wi-Fi connections.

Chapter \ref{host-scalability-chapter} now aims to investigate how well a multi-robot system can scale with ROS. As mentioned in Section \ref{background-scalability}, scalability can refer to many different concepts - in this chapter scalability will be considered in terms of vertical and horizontal scaling. Vertical scaling refers to running multiple identical processes on the same host machines. This has the opportunity to allow for more CPU cores to be utilised (in the case of a multi-core host), however can introduce issues if other system resources are a bottleneck (e.g. if there is not enough RAM to support the new processes, or the new processes are storage-heavy tasks causing resource contention). Horizontal scaling refers to distributing workloads across increasing numbers of host machines. This has the benefit that each host will generally have an entire set of resources (CPU, RAM, disk, GPU, etc) to itself, circumventing the issue of resource contention seen in vertical scaling. There are significant downsides to horizontal scaling though. Horizontal scaling introduces a need for complex communication between the hosts to organise and distribute tasks, or to communicate results. Horizontal scaling also has an economic impact, as it requires new hardware to be purchased and managed.

Full datasets and extra graphs from all experiments can be found in the experiment repository\cite{experimentGitHubRepo}.

\begin{table}[H]
    \centering
        \caption{Chapter \ref{host-scalability-chapter} Experiment Overview}
        \label{chapter-4-experiment-overview}
        \begin{tabular}{|p{0.25\columnwidth}|p{0.4\columnwidth}|p{0.1\columnwidth}|}
            \hline
            \textbf{Experiment}                & \textbf{Description}                                                                                                 & \textbf{Section}                       \\ \hline
            Vertical Scaling                   & Evaluating the ability of ROS to scale communication-heavy applications vertically (more nodes on the same hosts)    & \ref{experiment7-vertical-scaling}   \\ \hline
            Vertical Scaling on Car Platform   & Confirming the results of the previous experiment are applicable to the robot car kit platform                       & \ref{experiment8-vertical-scaling}   \\ \hline
            Horizontal Scaling on Car Platform & A proposal to evaluate the ability of ROS to scale communication-heavy applications horizontally (adding more hosts) & \ref{experiment9-horizontal-scaling} \\ \hline
        \end{tabular}
\end{table}

\subfile{sections/experiment7}

\subfile{sections/experiment8}

\subfile{sections/experiment9}



\subfile{sections/conclusion}

%%%%%%%%%%%%%%%%
%              %
%  APPENDICES  %
%              %
%%%%%%%%%%%%%%%%
\begin{appendices}

\chapter{Continued Middlewares Overview}
\label{middlewares-overview-appendix}
\subfile{sections/background-part2-middleware-overview-appendix}

\end{appendices}

%%%%%%%%%%%%%%%%%%%%
%   BIBLIOGRAPHY   %
%%%%%%%%%%%%%%%%%%%%

\bibliographystyle{plain}
\bibliography{bib}

\end{document}
