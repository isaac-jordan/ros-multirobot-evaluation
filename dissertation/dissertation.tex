\pdfoutput=1

\documentclass{l4proj}

%
% put any packages here
%
\usepackage{float}

\begin{document}
\title{On The Scalability of ROS}
\author{Isaac Jordan}
\date{\today}
\maketitle

\begin{abstract}
Robots, distributed systems, and middleware.
\end{abstract}

\educationalconsent
%
%NOTE: if you include the educationalconsent (above) and your project is graded an A then
%      it may be entered in the CS Hall of Fame
%
\tableofcontents
%==============================================================================

\chapter{Introduction}
\pagenumbering{arabic}

Robotics is a fast-progressing field which has seen major advances the past decades. From obvious examples such as Amazon's item pickers, to more integrated applications such as autonomous cars - the scale of robotics is increasing.



%\vspace{-7mm}
%\begin{figure}
%\centering
%\includegraphics[height=9.2cm,width=13.2cm]{uroboros.pdf}
%\vspace{-30mm}
%\caption{An alternative hierarchy of the algorithms.}
%\label{uroborus}
%\end{figure}

\chapter{Background}

\section{Robotics}

Robots are the future. (Why robots are important)

\section{Robotic Middleware}

Robot designers want to work at a high level. Sensors are low level. (Why is middleware needed, whats available)

\section{ROS (Robot Operating System)}

ROS is what I'll be testing out. (What is it)

\section{Configuration of Robots}

The robots used in this project are 9 identical robot cars with front-wheel steering. (What is their set up)


\chapter{Experiment 1}

The first experiment was designed to analyse the transfer time of messages between two machines at varying message frequencies. This would highlight whether there was some limit as to how often ROS could send and receive messages on it's topics.

These two machines were Raspberry Pi 3 Model Bs connected via ethernet to an Asus router.

The messages were sent and received using ROS Kinetic, running on the Raspbian OS.

The experimental setup was that one Raspberry Pi would run a ROS master node, another Raspberry Pi runs a sender program that notes down the message-sent time, and sends it to a 3rd Raspberry Pi which merely echoes the message back to the sender. Upon receiving the message, the sender/receiver notes the current time, and writes `message X which was sent at Y, was received at Z' to a text file. The resulting Round Trip Time (RTT) for each message would therefore be the difference between the message-sent time and the message-received time.

The expected result of the experiment was that message latency (RTT) would be the same across all lower frequencies, until some bottleneck was reached that would then cause message latencies to exponentially increase due to congestion.

Code had been written prior to execute this experiment, so initially this was used\cite{Experiment1InitialCode}. However, this gave results that were contrary to the hypothesis. An increase in message frequency resulted in a reduction in message latency. A number of messages on higher frequencies were also dropped, and never received. As this was the opposite of the hypothesis, the first step was to critique the experiment code.

This review highlighted two major issues, the first was the echoing machine had a delay similar to the sender when the experiment design mandated that the echoer always respond as fast as it can. The second issue was the the maximum message queue size in ROS (how many messages can be buffered at once to compensate for a slow subscriber) was set equal to the message frequency of that run.

These issues were resolved by removing the code that executed the delay in the echoer, and by setting the maximum queue size to be equal to 1000 in every experiment (the number of messages expected to be sent).

The experiment was then repeated using this code, and the results from these runs agreed with the hypothesis.

\chapter{Experiment 2}

Experiment 2 was an investigation in to whether the performance of ROS messages was affected by previous messages sent on the system. This was tested by comparing the performance of ROS messages of 5 consecutive message passing runs vs 5 message passing runs with system reboots between runs. This was repeated at 1KHz, 4KHz, 7KHz, and 10Khz frequencies for two reasons. The first was to investigate whether areas in which we observed consistent performance before would exhibit any difference between reboot and no reboot. Secondly, it would give further insight in to where the exact barrier between `good, consistent performance' and `poor, erratic performance' is.

Rebooting the systems involved has the opportunity to affect performance by stopping any background processes, interrupting slow processing messages from previous runs, and resetting any message caches and buffers in memory.

The result of the experiment was hypothesised to demonstrate no significant difference between rebooting and not-rebooting at any message frequency.

\begin{figure}[h]
\centering
\includegraphics[width=\textwidth]{images/no-reboot-1khz.png}
\caption{Experiment 2 - No Reboot 1KHz Message Frequency}
\label{exp2-noreboot-1khz}
\end{figure}

As Figure \ref{exp2-noreboot-1khz} shows, no reboot runs for a message frequency of 1KHz were consistently low across all runs, ~1.5ms for most messages and never peaking more than 5ms. Figure \ref{exp2-fullreboot-1khz} shows the corresponding graph for the full-reboot run which, except for run 2, was extremely similar - although with increased number of messages dropped at the start of every run.

Results were similar for all other frequencies, although all other frequencies exhibited the generally inconsistent performance that was observed for high-frequency message frequencies.

\begin{figure}[h]
\centering
\includegraphics[width=\textwidth]{images/full-reboot-1khz.png}
\caption{Experiment 2 - Full Reboot 1KHz Message Frequency}
\label{exp2-fullreboot-1khz}
\end{figure}

%%%%%%%%%%%%%%%%
%              %
%  APPENDICES  %
%              %
%%%%%%%%%%%%%%%%
\begin{appendices}

\chapter{Experiment 2 Other Graphs}

\begin{figure}
\centering
\includegraphics[width=\textwidth]{images/no-reboot-4khz.png}
\caption{Experiment 2 - No Reboot 4KHz Message Frequency}
\label{exp2-noreboot-4khz}
\end{figure}

\begin{figure}
\centering
\includegraphics[width=\textwidth]{images/full-reboot-4khz.png}
\caption{Experiment 2 - Full Reboot 4KHz Message Frequency}
\label{exp2-fullreboot-4khz}
\end{figure}

\begin{figure}
\centering
\includegraphics[width=\textwidth]{images/no-reboot-7khz.png}
\caption{Experiment 2 - No Reboot 7KHz Message Frequency}
\label{exp2-noreboot-7khz}
\end{figure}

\begin{figure}
\centering
\includegraphics[width=\textwidth]{images/full-reboot-7khz.png}
\caption{Experiment 2 - Full Reboot 7KHz Message Frequency}
\label{exp2-fullreboot-7khz}
\end{figure}

\begin{figure}
\centering
\includegraphics[width=\textwidth]{images/no-reboot-10khz.png}
\caption{Experiment 2 - No Reboot 10KHz Message Frequency}
\label{exp2-noreboot-10khz}
\end{figure}

\begin{figure}
\centering
\includegraphics[width=\textwidth]{images/full-reboot-10khz.png}
\caption{Experiment 2 - Full Reboot 10KHz Message Frequency}
\label{exp2-fullreboot-10khz}
\end{figure}

\chapter{Running the Programs}

To compile this dissertation:
\begin{verbatim}

	> pdflatex dissertation
	> bibtex dissertation
	> pdflatex dissertation
    > pdflatex dissertation

\end{verbatim}


An example of running from the command line is as follows:
\begin{verbatim}
	> java MaxClique BBMC1 brock200_1.clq 14400
\end{verbatim}
This will apply $BBMC$ with $style = 1$ to the first brock200 DIMACS instance allowing 14400 seconds of cpu time.

\chapter{Generating Random Graphs}
\label{sec:randomGraph}
We generate Erd\'{o}s-R\"{e}nyi random graphs $G(n,p)$ where $n$ is the number of vertices and
each edge is included in the graph with probability $p$ independent from every other edge. It produces
a random graph in DIMACS format with vertices numbered 1 to $n$ inclusive. It can be run from the command line as follows to produce
a clq file
\begin{verbatim}
	> java RandomGraph 100 0.9 > 100-90-00.clq
\end{verbatim}
\end{appendices}

%%%%%%%%%%%%%%%%%%%%
%   BIBLIOGRAPHY   %
%%%%%%%%%%%%%%%%%%%%

\bibliographystyle{plain}
\bibliography{bib}

\end{document}
